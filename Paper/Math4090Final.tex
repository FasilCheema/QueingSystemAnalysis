documentclass{article}
\usepackage[utf8]{inputenc}
\date{}

\begin{document}

\begin{center} \section*{A service system with on-demand agent invitations} \end{center}
\newline
\begin{center} Guodong Pang · Alexander L. Stolyar \end{center}
\vspace*{+3cm}
\newline \begin{center} \large Ajanthan Mathialagan 214949861 \end{center}
\newline \begin{center} \large Fasil Cheema 214427876 \end{center}
\newline \begin{center}\large  Marisa Signorile 214724108 \end{center}

\newpage

	\section{Summary}
Our project is concerned with a service system with servers/agents that are invited on demand. The system has two components: the difference between the queues of agents and customers and the number of pending invitations for agents. The model establishes that in the asymptotic regime, that is when the customer arrival rate goes to infinity and the agents’ response rate is fixed, both customer and agent waiting times vanish when the system reaches a steady state. In this model the service time itself is not considered, and the customer arrival is modelled via a Poisson arrival process, and the agent responses is an i.i.d exponential distribution. The two variables: the difference between the agent/customer queues and the number of pending invitations are modelled via a CTMC (Continuous Time Markov Chain). Due to the difficulty of modelling the process exactly the analysis is done in the asymptotic regime mentioned earlier, which is known as the many-server asymptotic regime. \newline\newline
The paper shows that when the customer arrival rate becomes constant the model converges to a fluid limit and uniform global stability of fluid limits. We will show the proof of this first theorem which deals with properties of fluid limits by considering fluid models, defined as locally Lipschitz continuous trajectories (y,x) satisfying the properties in Theorem 1. In other words, if a fluid limit exists (which we prove in this report) it is it is necessarily a fluid model. Then we will prove theorem 2 in the paper showing that the model exhibits process stochastic stability and the limit interchange property. In addition, we will prove that on the diffusion scale, which is represented by r 1/2 scale, where r is a scaling parameter, convergence goes to the diffusion limit process (theorem 3). In proving theorem 3 in the paper, what is particularly interesting is the existence of an independent standard Brownian motions $B_i$ , i = 1, 2, corresponding to the driving unit-rate Poisson processes $N_i$ , i = 1, 2, all constructed on the same probability space. What is obtained is the claimed distribution convergence of the limit diffusion process ($Y^r , X^r$) which follow directly from linear SDEs. By being able to model this agent/ customer queue by  a linear SDE, one turns a complex behaviour asymptotic queue, into a model that can be used to predict customer arrival rate approaching infinity. Finally, for the stochastic model tightness and limit-interchange results are proven for the model. The applications of Brownian motion in queuing theory is very useful to real world analyzing call centres and internet browsing queues. The limiting regime under the diffusion scaling is a reflected Brownian motion and leads to a diffusion approximation of the original queue model.\newline\newline
The simulation conducted demonstrates how the unique fluid limit trajectory provides a good approximation of the system dynamics. What is also shown is the behavior and performance of feedback agent invitation schemes, as well as accuracy of the approximations given by the theoretical results in theorems 1-4 from the paper. A note able conclusion from the study of service systems with on-demand agent invitations is that they demonstrate the desirable performance, that is, both customer and agent waiting delays vanish as the system scale increases to infinity.

	\section{Proofs of Theorems}
	\section*{Theorem 1}
Consider a sequence of processes $(Y^r , X^r), r \rightarrow \infty$, with deterministic initial states such that $(Y^r(0), X^(0))\rightarrow ( y(0) ,x(0))$ for some fixed $(y(0),x(0)) \in & \mathbb{R}^2, x(0) \geq −\lambda/ \beta $. Then, these processes can be constructed on a common probability space, so that the following holds. There exists a unique locally Lipschitz trajectory (y, x), such that, w.p.1,
\newline 

\begin{center} $(Y^r , X^r)\rightarrow (y,x)$ u.o.c. as $r\rightarrow \infty$
\end{center}

\newline where

\begin{center} $x(t)\geq −\lambda/ \beta t\geq 0$
\end{center}
and at any regular point $t \geq 0 $(all points $t\geq 0$ are regular, except a subset of zero Lebesgue measure), the following holds: if $x(t)> −\lambda/ \beta$ , 
\begin{center} 
\newline $y ^\prime (t) = \beta x(t)$,

\newline $x ^\prime (t) = −\gamma \beta x(t) - \varepsilon y(t)$, 
\end{center}
and if x(t) = − $\lambda/\beta$,
\begin{center} 
\newline $y ^\prime (t) = -\lambda$

\newline $x ^\prime (t) = [\gamma \lambda - \varepsilon y(t)] \vee 0$
\end{center}
The unique limit trajectory (y, x) will be called a fluid limit starting from (y(0), x(0)).
	
\section*{Proof of Theorem 1}

Given the initial state $(Y^r(0), X^r (0))$ in theorem 1, the processes will be proved for all r, that is, for $(Y^r , X^r)$:
\newline
\newline$Y^r(t)=Y^r(0) +N_2 (\beta \int_0^tX^r(s)ds)-N_1(\lambda rt) $
\newline
\newline $X^r(t)= Z^r(t) - (\min_{0\leq s\leq t} Z^r(t)) \vee 0$, since $X^r(t)$ cannot become negative.
\newline
\newline $Z^r(t)=X^r(0) + \gamma N_1(\lambda rt) - \gamma N_2(\beta \int_0^tX^r(s)ds) +N_3(\epsilon \int_0^t(Y^r(s))^-ds) + N_4(\epsilon \int_0^t(Y^r(s))^+ds)$
\newline
\newline where $N_i(\cdot)$ are independent unit rate Poisson processes for i=1,..,4.
\newline
\newline The functional law of large numbers holds for each Poisson process $N_i(\cdot)$:
\newline
\begin{center}$ N_i(rt)/r \rightarrow t,r \rightarrow \infty$, u.o.c, w.p.1
\end{center}
\newpage 
Define a fluid-scaled processes with centering as
\begin{center}
$(Y^r , X^r):= r^{-1} (Y^r, X^r - \lambda r/\beta)$
\end{center}
\newline
\newline
Let c be a constant, where $ c \geq \| y(0), x(0) \|$. Define $(Y_c^r,X_c^r)$ as a modified fluid scaled process following the same path as $(Y^r,X^r)$ until the first time that $\|(Y^r(t),X^r(t)\| \geq c$. Denote this time as $\tau_c^r$, then, at this time, the process halts at the value $(Y^r(\tau_c^r),X^r(\tau_c^r))$. 
\newline
\newline Next, we need to prove convergence for the fluid scaled process:
\newline (i) We must show the convergence of $(Y_c^r,X_c^r)$ to a limit trajectory that behaves like a fluid model as long as the state norm is away from c, constructed above.
\newline (ii) Choose c large enough so that the limit trajectory never reaches norm level c, proving it is a unique fluid model. This will imply that on any finite time interval, w.p.1, for all large r ,$(Y_c^r, X_c^r)$ coincides with $(Y^ r , X^ r )$, where $(Y^ r , X^ r )$ converges to the fluid model.
\newline\newline
For the modified fluid scaled process $(Y_c^r,X_c^r)$, the associated counting process for upward and downward jumps for $t\leq \tau _c^r$ are:
\newline
\newline $Y_c^r \uparrow(t) = r^{-1} N_2(r\beta \int_0^t[X^r_c(s) +\lambda/\beta]ds)$
 \newline $Y_c^r \downarrow(t) = r^{-1} N_1(\lambda rt)$
\newline  $X_c^r \uparrow(t) = r^{-1}\gamma N_1(\lambda rt) +r^{-1} N_3 (r \epsilon \int_0^t(Y^r_c(s))^- ds)$
\newline $X_c^r \downarrow(t) =r^{-1}\gamma N_2(r \beta \int_0^t[X^r_c(s) +\lambda/\beta]ds) +r^{-1} N_4 (r\epsilon\int_0^t(Y^r_c(s))^+ ds)$
\newline\newline
These counting processes are halted at time $\tau_c^r$. For 0\leq t \leq $\tau_c^r$, the original and modified processes $(Y^r , X^r)$ and $(Y_c^r,X_c^r)$ coincide. Thus,
\newline\newline
$Y^r_c(t)=Y^r(0) + Y_c^r \uparrow(t) - Y_c^r \downarrow(t)$
\newline $X^r_c(t)=Z^r_c(t) + (- \lambda/ \beta - min_{0\leq s \leq t} Z^r_c (s)) \vee 0$
\newline $Z^r_c(t)= X^r(0) + X_c^r \uparrow(t) - X_c^r \downarrow(t)$
\newline\newline
With the functional strong law of large numbers and the fact that the processes $Y^r_c$ and $X^r_c $ are uniformly bounded by
construction, we see that, w.p.1. for any subsequence of r , there exists a further sub- sequence along which the set of trajectories ($Y^r_c\uparrow,Y^r_c\downarrow,X^r_c\uparrow,X^r_c\downarrow)$ converges u.o.c. to a set of non-decreasing Lipschitz continuous functions ($y^r_c\uparrow,y^r_c\downarrow,x^r_c\uparrow,x^r_c\downarrow)$. Taking the limit:
\newline
$y^r_c\uparrow= \beta \int_0^t(x_c(s) + \lambda/\beta) ds)$
\newline $y^r_c\downarrow=\lambda t$
\newline $x^r_c\uparrow= \gamma \lambda t + \int_0^t y_c^-(s)ds$
\newline $x^r_c\downarrow= \gamma \beta \int_0^t (x_c(s) +\lambda/\beta) ds + \epsilon \int_0^t y_m^+(s) ds$
\newline\newline
 Along the chosen subsequence, u.o.c . convergence of $(Y^r,X^r$ to the fluid model (y ,x) holds. This means that w.p.1 the u.o.c.convergence of $(Y^r,X^r)$ to (y,x) holds for the original sequence. Thus, theorem 1 is proved. 

 \newpage
 \section*{Theorem 2}
 For all sufficiently large r, the system is stable, i.e., the Markov process $(Y^r, X^r)$ is positive recurrent. The sequence of stationary distributions of the fluid- scaled processes $(Y^r , X^ r) $ converges to the Dirac measure concentrated at (0, 0).
 
 \section*{Proof of Theorem 2}
 We use lemma 5 in the paper as an assumption for this proof. That is, for any initial state (y(0), x(0)), there is a unique fluid model starting from it. Moreover, uniformly on the initial states from a given compact set, \newline $(y(t), x(t))$ \rightarrow (0, 0), 
  $t \rightarrow \infty$  and $max_{t\geq 0}$ $\|(y(t),x(t))\|$ is bounded.
 \newline\newline Thus, we need to prove that For all sufficiently large r, the process $(X^r,Y^r)$ is stable, with a unique stationary distribution and the sequence of stationary distributions of $(X^r,Y^r)$ is tight.
 \newline\newline
 Define s(t) = (Y(t), X(t)) as the random process. Next,  consider the embedded Markov chain with fixed constants $\delta > 0$ and $\tau_ {max} > 0$. For the process starting from a given state s = s(0), consider the random stopping time $ \tau _\delta $(s), which is the first time t when $| \|s(t)\| - \|s\| | \geq \delta $; we then define the stopping time $\tau$ (s)= $\tau _{\delta}$(s)∧ $\tau_{max}$. Define a sequence of stopping times $\tau^{(k)}$ , k =1,2,... by 
	
\newline \newline $\tau(1)$ = $\tau$(s(0)), …
 \newline $\tau$ (k) = [$\tau^{k - 1}$ + $\tau^{max}$] $\land $ inf($t >$ $\tau^{k - 1}$ : $\mid \|s(t)\|_\ast$ − \|s($\tau^{k-1})\|_\ast$ $\mid \geq \delta$), \newline for k = 2, 3, ... 
 \newline\newline
 Consider the embedded discrete-time Markov chain s(k), k = 0, 1, ..., using $\tau$(k) as sampling times. Specifically, if s(t), $t \geq 0$, is the original continuous-time Markov process, then: 
 \newline
 s\hat(0) = s(0), s\hat(k) = s($\tau$(k)), k = 1,2,.... 
 \newline \newline Let $\phi$(s) = $\|s\|_\ast$. For the embedded chain $s\hat$, we show that, for some $C_1, C_2  > 0$,  uniformly in r, 
 \newline
 \begin{center} \mathbb{E}$[\phi^2(s \hat(1)) - \phi^2(s\hat(0)) \mid s \hat(0)] \leq  -  C_1 \phi(s \hat(0)) + C_2$. \end{center} \newline
 For some constant $\delta_7 > 0$, for any sequence r $\rightarrow \infty$ and corresponding s\hat(0) = $s\hat^r(0) $ such that $\|s\hat^r (0)\|\ast \uparrow \infty $, we have 
\begin{center} $\mathbf{P}[\phi(s ^r(1)) - \phi(s ^r (0)) \leq - \delta_7] \rightarrow  1 $ \end{center}
\newline It suffices to consider a sequence such that the convergence 
 $  \frac{1} {\|s ^r (0)\|_ \ast} s ^r ( 0 ) \rightarrow s $ holds, for some vector s with $\|s\|_\ast = 1.$
 \newline
We will study the behavior of the continuous-time process s(t), with initial state s(0) = $s\hat(0)$, on the interval [0, $\tau$(s(0))]. 
\newpage
\newline For any vector s = (y,x) we denote $s\prime = (y\prime, x\prime)$, where $y\prime = $\beta$ x, and  \newline x\prime = - \varepsilon y -\gamma \beta x $
Similarly, let $\|s\|_\ast$ denote (d/dt)$\|s(t)\|_\ast$ when s(t) = s. 
\newline Suppose that s=(y,x) and $x > 0$, then, (the sequence of processes can be constructed on a common probability space, such that) w.p.1, u.o.c: \newline \begin{center}
$s(t/\|s(0)\|_\ast) - s(0)\rightarrow s\prime t $ and $ \|s(t/\|s(0)\|_\ast)\|_\ast - \| s(0)\|\ast $ \rightarrow $\|s\|\prime_\ast t $. \end{center} 
\newline We see that $\tau$(s(0)) = $\tau _\delta$ (s(0)) $\rightarrow 0$, and therefore, \begin{center} $\mathbf{P}[\phi(s ^r(1)) - \phi(s ^r (0)) \leq - \delta_7] \rightarrow  1 $ holds with $\delta_7$ = $\delta$.\end{center}	
Consider the sub-case when $[x(0) - (-\lambda / \beta)]/\mid x \prime \mid \rightarrow \infty $; 
\newline \newline We check that $s(t/\|s(0)\|_\ast) - s(0)\rightarrow s\prime t $ and $ \|s(t/\|s(0)\|_\ast)\|_\ast - \| s(0)\|_\ast $ \rightarrow $\|s\|\prime_\ast t $ still holds. This is the scenario when the time $\tau _{hit}$ for the x(t) to hit boundary $-\lambda/ \beta$ is such that $\tau_{hit} \rightarrow 0$ and $\tau_{hit} \|s(0)\|_\ast \rightarrow \infty $ therefore,$ \|s(t)\|_\ast$, decreases by $\delta$ before time $\tau_{hit}$, and $\mathbf{P}[\phi(s ^r(1)) - \phi(s ^r (0)) \leq - \delta_7] \rightarrow  1 $ follows.
\newline \newline Finally, consider the sub-case when [x (0) -$(-\lambda/ \beta)] /\mid x\prime \mid \rightarrow c  \epsilon [ 0 , \infty )$ . 

\newline In this sub-case, $\tau_{hit} \|s(0)\|_\ast \rightarrow c.$ Then, we consider the process such that in the interval [0, $\tau_{hit} \|s(0)\|_\ast$] it is the process with time slowdown, as in  $s(t/ \|s(0)\|_\ast) - s(0) \rightarrow s \prime t $ and $\|s(t/\|s(0)\| _\ast∗) - \|s(0)\|_\ast \rightarrow  \|s\| _\ast t.$ From time $\tau_{hit}\|s(0)\|_\ast$ to infinity, the process continues in actual time, without slowing down. W.p.1. in the limit we obtain the trajectory which satisfies\newline $s(t/\|s(0)\|_\ast) − s(0) \rightarrow s\prime t $ and $\|s(t/\|s(0)\|_\ast)\| - \|s(0)\|_\ast \rightarrow \|s\|_\ast t $ in the interval [0, c], and then in the interval [c,$\infty$) we have x(t) = $-\lambda/\beta $ and y \prime(t) = $-\lambda$. In both intervals, the limit trajectory is such that the norm $\|s(t)\|_\ast$ is decreasing at least at some positive rate. \newline\newline
From \mathbb{E}$[\phi^2(s \hat(1)) - \phi^2(s\hat(0)) \mid s \hat(0)] \leq  -  C_1 \phi(s \hat(0)) + C_2$ we conclude that the embedded chain is stable for each sufficiently large r , and therefore has stationary distribution which is easily seen to be unique. Moreover, the stationary distributions are such that, uniformly in (sufficiently large) r, \mathbb{E}$\phi( s\hat(\infty) \leq C_2/C_1$.
\newline \newline We also observe that, for any fixed $C_3 > 0$, uniformly on all $\|s\|_\ast \leq C_3$ and all r, $\mathbb{E} \tau $(s)$ \geq C_4 > 0$ . Let us choose $C_3$ large enough, so that for the embedded chain in steady-state, $\mathbf{P}{\|s\hat(\infty)\|_\ast\leq C_3}\geq 1/2 $.
\newline \newline
Now we use the relation between stationary distributions of the original continuous- time process and the sampled chain: \newline
$\mathbf{P}[s(\infty) \epsilon A] $ = $\mathbf{E}[ $\mathbf{E} [\int_{0}^{\tau(s(0)} I[s(t)\epsilon A dt \mid s(0)=s\hat(\infty)]]]/ [$\mathbf{E}[\tau(s\hat(\infty)$)]
\newline 
Then we see that our original continuous-time process is stable for each sufficiently large r, and the stationary distributions are such that, uniformly in (sufficiently large) r, we have: \newline

$\mathbf{E} \|s(\infty)\| _\ast \leq \frac {\mathbf{E} \|s\hat\| _\ast + 2\delta] \tau_{max}} {C_4/2} \geq C_5 \mathbf{E}\|s(\infty)\|_\ast + C_6 \leq C_7.$
\newline Therefore, the uniform bound on the expected norm in steady-state implies the tightness of stationary distributions.  Theorem 2 is thus, proved.	

\section{Theorem 3:}
Suppose there exists a sequence of deterministic initial states, such that 
$(\hat{Y}^{r}(0), \hat{X}^{r}(0)) \rightarrow{} (\hat{Y}(0), \hat{X}(0))$, where $(\hat{Y}(0), \hat{X}(0))$ is fixed in $\mathbb{R}^2$. Then, 
$$(\hat{Y}^{r}, \hat{X}^{r}) \implies (\hat{Y}, \hat{X}) $$
Where $(\hat{Y}, \hat{X})$ are unique solutions to the stochastic differential equations. 
$$\\$$

$$\boldsymbol{Proof}:$$ 

We have $(\hat{Y}^{r}(0), \hat{X}^{r}(0)) \rightarrow{} (\hat{Y}(0), \hat{X}(0))$. In addition we know that $(\Bar{Y}^r,\Bar{X}^r)$ are unique solutions to the fluid scaled process such that $(\Bar{Y}^r(0),\Bar{X}^r(0)) \rightarrow{} (0,0)$. Applying Theorem 1 we have that, $(\Bar{Y}^r(t),\Bar{X}^r(t))$ converges uniformly on compact  to the fluid limit which is $((y(t),x(t)) = (0,0)$ as
$r \rightarrow \infty$ 
Hence, with the probability of 1.
$$ \Bar{Y}^r \rightarrow 0, \Bar{X}^r \rightarrow{} 0$$
gives us that, 
$$\int_{0}^{t}\abs{\Bar{Y}^r(s)}ds \rightarrow{} 0$$ $$\int_{0}^{t}\abs{\Bar{X}^r(s)}ds \rightarrow{} 0$$ 
which are uniformly on compact. This implies that, with the probability of 1 on a finite time set $[0,T]$, it is not hard to reach The boundary of $\hat{X}^r$ which is sufficiently large $r$ For instance,  $\hat{X}^r > -\lambda \frac{\sqrt{r}}{\beta}$
Using the representations of $Y^r(t)$ and $X^r(t)$ in theorem 1 and We have from theorem 1 that, 

$$Y^r(t) = Y^r(0) + N_2(\beta \int_{0}^{t}X^r(s)ds) - N_1(\lambda r t)$$
$$X^r(t) = Z^r + (-\min_{0 \leq s \leq t} Z^r(s) ) \ 0$$
$$Z^r(t) = X^r(0) + \gamma N_1(\lambda rt) - \gamma N_2(\beta \int_{0}^t X^r(s)ds) + N_3(\epsilon \int_0^t(Y^r(s))^{-}ds) - N_4(\epsilon \int_0^t(Y^r(s))^{+}ds)$$

applying aspects of Lemma 9 which states. A unit rate Poisson Process $\prod (t) : t \geq 0$ can be realized as some probability space as a standard Brownian motion $\beta (t): t \geq 0$, such that there exists a random positive variable $\epsilon$ which represents a finite moment generating function in a neighbourhood of the origin. In addition holding that $Z^r = X^r$ in the representations from theorem 1. We finally get, 
$$\hat{Y}^r(t) = Y^r(0) + \beta \int_{0}^{t}X^r(s)ds + \beta_2(\lamda t) - \beta_1(\lamda t) + \delta_1^r(t)$$
$$\hat{X}^r(t) = \hat{X}^r(0) - \gamma \int_{0}^t \beta \hat{X}^r(s)ds - \epsilon \int_{0}^t\hat{Y}^r(s)ds + \gamma \beta_1(\lamda t) + \gamma\beta_2(\lamda t) + \delta_2^r(t)$$
where $\delta_i^r(t)$ is a constant for $i =1,2$
By letting $r\rightarrow{} \infty$ and applying Lemma 10 which states the mapping of a certain number $\psi$ is continuous in the topology of convergence uniformly on compact. We obtain a probability of 1. Thus the convergence which is uniformly on compact of $(\hat{Y}^r , \hat{X}^r)$ to a limiting distribution process $(\hat{Y},\hat{X})$. Thus, satisfies the convergence condition of theorem 3. 	
	
	
\end{document}
